\documentclass[11pt]{article}

\usepackage[portuguese]{babel}
\usepackage[utf8]{inputenc}
\usepackage{amsmath}
\usepackage{graphicx}
\usepackage{float}
\usepackage{subfig}
\usepackage{fixltx2e}
\usepackage[bottom]{footmisc}
\usepackage{color}
\usepackage{todonotes}
% \usepackage[usenames,dvipsnames]{xcolor}
\usepackage[font=footnotesize]{caption}

\usepackage{titlesec}
\setcounter{secnumdepth}{4}
\titleformat{\paragraph}
{\normalfont\normalsize\bfseries}{\theparagraph}{1em}{}
\titlespacing*{\paragraph}
{0pt}{3.25ex plus 1ex minus .2ex}{1.5ex plus .2ex}

\numberwithin{equation}{section}

\linespread{1.3}
\usepackage{indentfirst}
\usepackage[top=2cm, bottom=2cm, right=2.2cm, left=2.2cm]{geometry}
\addto\captionsportuguese{\renewcommand{\contentsname}{Índice}}

\begin{document}

\begin{titlepage}
\begin{center}

\hfill \break
\hfill \break

\includegraphics[width=0.3\textwidth]{./logo}~\\[1cm]

\textsc{\LARGE Instituto Superior Técnico}\\[0.25cm]
\textsc{\Large Mestrado Integrado em Engenharia Electrotécnica e de Computadores}\\[1.8cm]
\textsc{\huge Arquitecturas Avançadas de Computadores}\\[0.25cm]

{\huge \bfseries Descrição do processador $\mu$Risc a funcionar em pipeline\\[1.2cm]}

\begin{tabular}{ l l }
Guilherme Branco Teixeira & \hspace{2mm} n.º 70214 \\ 
Maria Margarida Dias dos Reis & \hspace{2mm} n.º 73099 \\
Nuno Miguel Rodrigues Machado & \hspace{2mm} n.º 74236 
\end{tabular}

\vfill

{\large Lisboa, 7 de Maio 2014} 

\end{center}
\end{titlepage}
 
\pagenumbering{gobble}
\clearpage

\footnote{É de referir que as imagens e tabelas não foram colocadas em anexo de modo a permitir uma melhor compreensão do relatório.}

\tableofcontents
\pagebreak

\clearpage
\pagenumbering{arabic}

\section{Introdução}

Com este trabalho laboratorial pretende-se projectar um processador $\mu$Risc com funcionamento \textit{pipelining}. O processador possui 4 andares de \textit{pipelining}, no primeiro andar é feito o \textit{instruction fetch} (IF), no segundo andar é feito o \textit{instruction decode} (ID) e o \textit{operand fetch} (OF), no terceiro andar são executadas operações da ALU (EX) e de acesso à memória de dados (MEM) e, por fim, no quarto é feita a escrita no banco de registos, o \textit{write back }(WB). Com o funcionamento em \textit{pipelining} poderá correr dois tipos de conflitos, dados \textit{(data hazards)} e de controlo \textit{(control hazards)}. 
 

\section{Métodos de resolução para dependências e conflitos}


Será apresentado em primeiro lugar, os métodos e técnicas de resolução dos conflitos de controlo e dados. E em segundo lugar quais as técnicas utilizadas.


\subsection{Conflitos de dados, \textit{data hazards}}
Conflito que surge quando uma instrução depende dos	resultados de uma instrução anterior, de forma a afectar o resultado obtido pela linha de processamento.

\begin{itemize}
	\item \textbf{Solução 1}: Bloqueio dos andares do pipeline, \textit{stall}, até que os dados correctos estejam disponíveis;
	\item \textbf{Solução 2}: Se o dado correcto existir algures no pipeline, estabelece-se um bypass para o andar correcto, aplica-se a técnica de \textit{forwarding};
	\vspace{-2.5mm}
	\item \textbf{Solução 3}: Escalonar/reordenar as instruções, se a ordenação for feita pelo compilador, tem-se um escalonamento estático, se for feita pelo \textit{hardware}, escalonamento dinâmico;
\end{itemize}

\subsection{Conflitos de controlo, \textit{control hazards}}
Conflito que surge quando uma instrução de controlo condicional depende dos	resultados de uma instrução anterior, de uma forma a impedir um predição correcta.

\begin{itemize}
	\item \textbf{Solução 1}: BTB, foi contruída uma BTB com 9 bits de largura e apenas um bit de predição (BPB);
	\item \textbf{Solução 2}: \textit{forwarding} de flags, após se obter o resultado necessário para a predição, establece-se um bypass para verificar a condição do salto;
	\vspace{-2.5mm}
\end{itemize}

\subsection{Conflitos de controlo, \textit{control hazards}}

\section{Estrutura do Processador}

\section{Conclusões}

\pagebreak

\listoftodos

\end{document}