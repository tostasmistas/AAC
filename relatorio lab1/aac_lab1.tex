\documentclass[11pt]{article}

\usepackage[portuguese]{babel}
\usepackage[utf8]{inputenc}
\usepackage{amsmath}
\usepackage{graphicx}
\usepackage{float}
\usepackage{subfig}
\usepackage{fixltx2e}
\usepackage[bottom]{footmisc}
\usepackage{color}
\usepackage{todonotes}
% \usepackage[usenames,dvipsnames]{xcolor}
\usepackage[font=footnotesize]{caption}
\numberwithin{equation}{section}

\linespread{1.3}
\usepackage{indentfirst}
\usepackage[top=2cm, bottom=2cm, right=2.5cm, left=2.5cm]{geometry}
\addto\captionsportuguese{\renewcommand{\contentsname}{Índice}}

\begin{document}

\begin{titlepage}
\begin{center}

\hfill \break
\hfill \break

\includegraphics[width=0.3\textwidth]{./logo}~\\[1cm]

\textsc{\LARGE Instituto Superior Técnico}\\[0.25cm]
\textsc{\Large Mestrado Integrado em Engenharia Electrotécnica e de Computadores}\\[1.8cm]
\textsc{\huge Arquitecturas Avançadas de Computadores}\\[0.25cm]

{\huge \bfseries Simulação de um processador $\mu$Risc\\[1.2cm]}

\begin{tabular}{ l l }
Maria Margarida Dias dos Reis & \hspace{2mm} n.º 73099 \\
Nuno Miguel Rodrigues Machado & \hspace{2mm} n.º 74236 
\end{tabular}

\vfill

{\large Lisboa, 29 de Março 2014} 

\end{center}
\end{titlepage}

\pagenumbering{gobble}
\clearpage

\tableofcontents
\pagebreak

\clearpage
\pagenumbering{arabic}

\section{Introdução}

Com este trabalho laboratorial pretende-se projectar um processador $\mu$Risc, de 16 \textit{bits} com arquitectura RISC. O processador possui 8 registos de uso geral e 42 instruções. O projecto do processador é feito com recurso a uma linguagem de descrição de \textit{hardware} - VHDL.

\section{Características do Processador}

O processador elaborado foi simulado para uma placa Artix 7 e tem as seguintes características:

\vspace{-2mm}

\begin{itemize}
  \item 16 \textit{bits};
  \vspace{-2.5mm}
  \item 8 registos de uso geral de 16 \textit{bits} de largura (R0, \ldots, R7);
  \vspace{-2.5mm}
  \item 42 instruções;
  \vspace{-2.5mm}
  \item instruções de 3 operandos;
  \vspace{-2.5mm}
  \item organização de dados na memória do tipo \textit{big endian};
  \vspace{-2.5mm}
  \item uma memória ROM de 8 KBytes (4k endereços $\times$ 2 \textit{bytes}) endereçada com palavras de 12 \textit{bits} utilizada para as instruções/programa e uma memória RAM de 8 KBytes (4k endereços $\times$ 2 \textit{bytes}) endereçada com palavras de 12 \textit{bits} para os dados.
  
\end{itemize}

\section{Estrutura do Processador}

O processador $\mu$Risc que foi projectado encontra-se dividido em quatro andares - num primeiro andar é feito o \textit{instruction fetch} (IF), no segundo andar é feito o \textit{instruction decode} (ID) e o \textit{operand fetch} (OF), no terceiro andar são executadas operações da ALU (EX) e de acesso à memória de dados (MEM) e, por fim, no quarto e último andar é feita a escrita no banco de registos, o \textit{write back} (WB).

\subsection{Primeiro Andar - IF}

No primeiro andar a próxima instrução a ser executada é carregada da memória ROM. 

\subsection{Segundo Andar - ID e OF}
\subsection{Terceiro Andar - EX e MEM}
\subsection{Quarto Andar - WB}

No último andar os resultados são escritos no banco de registos, sendo que os resultados possíveis de ser escritos são: 

\pagebreak

\listoftodos

\end{document}