\documentclass[11pt]{article}

\usepackage[portuguese]{babel}
\usepackage[utf8]{inputenc}
\usepackage{amsmath}
\usepackage{graphicx}
\usepackage{float}
\usepackage{subfig}
\usepackage{fixltx2e}
\usepackage[bottom]{footmisc}
\usepackage{listings}
\usepackage{color} 
\usepackage[usenames,dvipsnames]{xcolor}
\usepackage[colorinlistoftodos]{todonotes}
\usepackage[font=footnotesize]{caption}

\setcounter{tocdepth}{3}

\numberwithin{equation}{section}

\linespread{1.3}
\usepackage{indentfirst}
\usepackage[top=2cm, bottom=2cm, right=2.3cm, left=2.3cm]{geometry}
\addto\captionsportuguese{\renewcommand{\contentsname}{Índice}}

\begin{document}

\begin{titlepage}
\begin{center}

\hfill \break
\hfill \break

\includegraphics[width=0.3\textwidth]{./logo}~\\[1cm] 

\textsc{\LARGE Instituto Superior Técnico}\\[0.25cm]
\textsc{\Large Mestrado Integrado em Engenharia Electrotécnica e de Computadores}\\[1.8cm]
\textsc{\huge Electrónica Rápida}\\[0.25cm]

\vspace{6mm}

{\huge \bfseries Projecto e Simulação de Amplificadores Lineares para Altas Frequências\\[1cm]}

\begin{tabular}{ l l }
Guilherme Branco Teixeira & \hspace{2mm} n.º 70214 \\
Maria Margarida Dias dos Reis & \hspace{2mm} n.º 73099 \\
Nuno Miguel Rodrigues Machado & \hspace{2mm} n.º 74236
\end{tabular}

\vspace{7mm}

Grupo n.º 2 de quarta-feira das 11h00 - 12h30

\vfill

{\large Lisboa, 25 de Abril de 2015} 

\end{center}
\end{titlepage}

\pagenumbering{gobble}
\clearpage

\tableofcontents
\pagebreak

\clearpage
\pagenumbering{arabic}

\section{Introdução}

\section{Plano de Trabalhos}

\subsection{Projecto de um amplificador uniandar com linhas ideais}

\subsubsection{a) Projecto do amplificador}

\paragraph{1.}

ola

\paragraph{2.}
\paragraph{3.}
\paragraph{4.}

\subsubsection{b) Projecto do amplificador utilizando tecnologia microfita}

\subsection{Concretização do amplificador em tecnologia de microfita}

\subsubsection{a) Introdução de elementos que simulam descontinuidades nas linhas}

\subsubsection{b) Substituição do transístor e condensadores}

\section{Conclusões}

\end{document}