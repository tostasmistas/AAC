\documentclass[11pt]{article}

\usepackage[portuguese]{babel}
\usepackage[utf8]{inputenc}
\usepackage{amsmath}
\usepackage{graphicx}
\usepackage{float}
\usepackage{subfig}
\usepackage{fixltx2e}
\usepackage[bottom]{footmisc}
\usepackage{listings}
\usepackage{color} 
\usepackage[usenames,dvipsnames]{xcolor}
\usepackage[colorinlistoftodos]{todonotes}
\usepackage[font=footnotesize]{caption}

\setcounter{tocdepth}{3}

\numberwithin{equation}{section}

\linespread{1.3}
\usepackage{indentfirst}
\usepackage[top=2cm, bottom=2cm, right=2.3cm, left=2.3cm]{geometry}
\addto\captionsportuguese{\renewcommand{\contentsname}{Índice}}

\begin{document}

\begin{titlepage}
\begin{center}

\hfill \break
\hfill \break

\includegraphics[width=0.3\textwidth]{./logo}~\\[1cm] 

\textsc{\LARGE Instituto Superior Técnico}\\[0.25cm]
\textsc{\Large Mestrado Integrado em Engenharia Electrotécnica e de Computadores}\\[1.8cm]
\textsc{\huge Electrónica Rápida}\\[0.25cm]

\vspace{6mm}

{\huge \bfseries Projecto e Simulação de Amplificadores Lineares para Altas Frequências\\[1cm]}

\begin{tabular}{ l l }
Guilherme Branco Teixeira & \hspace{2mm} n.º 70214 \\
Maria Margarida Dias dos Reis & \hspace{2mm} n.º 73099 \\
Nuno Miguel Rodrigues Machado & \hspace{2mm} n.º 74236
\end{tabular}

\vspace{7mm}

Grupo n.º 2 de quarta-feira das 11h00 - 12h30

\vfill

{\large Lisboa, 25 de Abril de 2015} 

\end{center}
\end{titlepage}

\pagenumbering{gobble}
\clearpage

\tableofcontents
\pagebreak

\clearpage
\pagenumbering{arabic}

\section{Introdução}

O objectivo deste laboratório é estudar técnicas de projecto de amplificadores lineares de alta frequência, análise das suas características (estabilidade, ganho, adaptação e factor de ruído) e comportamentos. A caracterização dos dispositivos do amplificador será realizada através dos pârametros distribuídos - parâmetros S.

Utiliza-se um transístor da Hewlett-Packard (HP) ATF-35176, um transístor que utiliza tecnologia PHEMT (\textit{Pseudomorphic High Mobility Transistor}), preparado para trabalhar em altas frequências.

\section{Plano de Trabalhos}

As especificações do amplificador a construir podem ser consultadas na tabela seguinte, tal como as características do substrato plástico para alta-frequência da Taconic (TLY -3-0310-CH/CH), sobre qual o transístor irá ser implantado. 

\begin{table}[H]
	\centering
	\caption{Características do amplificador a projectar.}
	\vspace{-1.5mm}
	\includegraphics[keepaspectratio=true, scale=0.45]{teoricas/table1}
\end{table}

De notar que o valor da espessura do substrato foi modificado de 0.78 mm para 0.35 mm com o objectivo de garantir propagação transversal nas linhas de microfita, ou seja, garantir que estas têm um comprimento maior que a largura. 

Numa primeira fase do trabalho laboratorial é projectado e simulado o amplificador uniandar com linhas simétricas. Na segunda fase o amplificador é projectado com tecnologia de microfita.

\subsection{Projecto de um amplificador uniandar}

\subsubsection{a) Projecto do amplificador com linhas ideais}

Nesta primeira fase, o amplificador irá ser constituido pelo transistor descrito anteriormente, no entanto, todos os dispositivos utilizados no seu projecto e simulação serão dispositivos ideais, esta fase inicial tem como objectivo definir os parâmetros do amplificador que permitem obter as especificações pedidas.

\paragraph{PFR Pretendido:}

Em primeiro lugar, é feita uma análise DC ao transístor que tem em vista obter o ponto de funcionamento em repouso (PFR) especificado. O circuito que nos permitiu alcançar essa análise é o que se vê na Figura \ref{fig:Circuito_0}.

\begin{figure}[H]
	\centering
	\includegraphics[keepaspectratio=true, scale=0.42]{exps/Circuito_0}
	\vspace{-0.5em}
	\caption{Circuito utilizado para obter o PFR desejado.}
	\label{fig:Circuito_0}
	\vspace{-0.8em}
\end{figure} 

A análise DC serve para descobrir o valor de $ V_{GS} $ correspondente ao PFR desejado. No circuito da Figura \ref{fig:Circuito_0} existe um componente denominado de \texttt{I\_Probe} que tem como objectivo controlar o valor de $ I_{D} $ à medida que o valor de $ V_{GS} $ varia. Um excerto dos resultados desta análise pode ser consultado na Figura \ref{fig:VGS}, onde se pode concluir que o valor da tensão  $V_{GS}$ que melhor corresponde a uma corrente $I_{D}$ de $20$ mA ($20.03$ mA) é de $-0.277$ V.

\begin{figure}[H]
	\centering
	\includegraphics[keepaspectratio=true, scale=0.30]{exps/Vgs}
	\vspace{-0.5em}
	\caption{Valores de $V_{GS}$ correspondentes à corrente de \texttt{I\_Probe}.}
	\label{fig:VGS}
	\vspace{-0.8em}
\end{figure} 

\paragraph{Análise em Alta-Frequência:}

Com o transístor a funcionar no PFR desejado, é preciso construir um novo circuito que contenha condensadores e bobines ideais, \texttt{DC\_Block} e \texttt{DC\_Feed}, respectivamente, para que seja possível realizar a simulação dos parâmetros S. Este circuito apresenta-se de seguida.

\begin{figure}[H]
	\centering
	\includegraphics[keepaspectratio=true, scale=0.42]{exps/Circuito_1}
	\vspace{-0.5em}
	\caption{Circuito utilizado para obter o valores dos parâmetros S.}
	\label{fig:Circuito_1}
	\vspace{-0.8em}
\end{figure}

Simulando o circuito anteriormente projectado foram obtidos os seguintes valores para os parâmetros S, K (parâmetro de estabilidade), MAG (ganho de transdução máximo) e para as cargas de adaptação para o transístor à frequência central.

\begin{table}[H]
 	\centering
 	\caption{Parâmetros que definem o transístor.}
 	\vspace{-1.5mm}
 	\includegraphics[keepaspectratio=true, scale=0.45]{teoricas/table2}
\end{table}

De notar que os valores obtidos experimentalmente para os parâmetros S não podem ser verificados na \textit{datasheet} do transístor, uma vez que esta apenas especifica o comportamento do ATF-35176 para frequências entre 2 GHz e 18 GHz.

Com os valores da Tabela 2 determinados pode-se calcular o valor de $\Delta$, ou seja, o determinante da matriz de dispersão:

\vspace{-3mm}
\begin{equation}
\Delta = S_{11}S_{22} - S_{21}S_{12} = 0.069~/-6.83 ^{\circ}.
\label{eq:delta}
\end{equation}

\vspace{1mm} 
Como se pode ver, K $= 1.235 > 1$, $\lvert \Delta \rvert = 0.069 < 1$ e $\lvert S_{ii} \rvert < 1$, pelo que o transístor é incondicionalmente estável.

\paragraph{Projecção da Malha de Entrada e de Saída:}
\paragraph{4.}

\subsubsection{b) Projecto do amplificador utilizando tecnologia microfita}

\subsection{Concretização do amplificador em tecnologia de microfita}

\subsubsection{a) Introdução de elementos que simulam descontinuidades nas linhas}

\subsubsection{b) Substituição do transístor e condensadores}

\section{Conclusões}

\end{document}