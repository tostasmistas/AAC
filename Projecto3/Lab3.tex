\documentclass[11pt]{article}

\usepackage[portuguese]{babel}
\usepackage[utf8]{inputenc}
\usepackage{amsmath}
\usepackage{graphicx}
\usepackage{float}
\usepackage{subfig}
\usepackage{fixltx2e}
\usepackage[bottom]{footmisc}
\usepackage{color}
\usepackage{xargs}                      % Use more than one optional parameter in a new commands
\usepackage[pdftex,dvipsnames]{xcolor}  % Coloured text etc.
\usepackage[colorinlistoftodos,prependcaption,textsize=tiny]{todonotes}
\usepackage{listings}
\usepackage[font=footnotesize]{caption}

\definecolor{keywordcolor}{rgb}{0,0.4,0.7}
\definecolor{commentcolor}{rgb}{0.4,0.4,0.4} 	
\definecolor{mygray}{rgb}{0.5,0.5,0.5} 	% line counter color
\definecolor{mymauve}{rgb}{0.90,0.25,0.47}	% string color
\definecolor{codebackground}{rgb}{0.95,0.95,0.95} 

\newcommandx{\unsure}[2][1=]{\todo[linecolor=red,backgroundcolor=red!25,bordercolor=red,#1]{#2}}
\newcommandx{\change}[2][1=]{\todo[linecolor=blue,backgroundcolor=blue!25,bordercolor=blue,#1]{#2}}
\newcommandx{\info}[2][1=]{\todo[linecolor=OliveGreen,backgroundcolor=OliveGreen!25,bordercolor=OliveGreen,#1]{#2}}
\newcommandx{\improvement}[2][1=]{\todo[linecolor=Plum,backgroundcolor=Plum!25,bordercolor=Plum,#1]{#2}}
\newcommandx{\thiswillnotshow}[2][1=]{\todo[disable,#1]{#2}}
\usepackage[font=footnotesize]{caption}

\numberwithin{equation}{section}

\linespread{1.3}
\usepackage{indentfirst}
\usepackage[top=2cm, bottom=2cm, right=2.25cm, left=2.25cm]{geometry}
\addto\captionsportuguese{\renewcommand{\contentsname}{Índice}}
\lstset{ %
	backgroundcolor=\color{codebackground},   % choose the background color; you must add \usepackage{color} or \usepackage{xcolor}
	basicstyle=\ttfamily \footnotesize,        % the size of the fonts that are used for the code
	breakatwhitespace=false,         % sets if automatic breaks should only happen at whitespace
	breaklines=true,                 % sets automatic line breaking
	captionpos=b,                    % sets the caption-position to bottom
	commentstyle=\color{commentcolor},    % comment style
	deletekeywords={...},            % if you want to delete keywords from the given language
	escapeinside={\%*}{*)},          % if you want to add LaTeX within your code
	extendedchars=true,              % lets you use non-ASCII characters; for 8-bits encodings only, does not work with UTF-8
	keepspaces=true,                 % keeps spaces in text, useful for keeping indentation of code (possibly needs columns=flexible)
	keywordstyle=\color{keywordcolor},       % keyword style
	numbers=left,                    % where to put the line-numbers; possible values are (none, left, right)
	numbersep=5pt,                   % how far the line-numbers are from the code
	numberstyle=\tiny\color{mygray}, % the style that is used for the line-numbers
	rulecolor=\color{black},         % if not set, the frame-color may be changed on line-breaks within not-black text (e.g. comments (green here))
	showspaces=false,                % show spaces everywhere adding particular underscores; it overrides 'showstringspaces'
	showstringspaces=false,          % underline spaces within strings only
	showtabs=false,                  % show tabs within strings adding particular underscores
	stepnumber=1,                    % the step between two line-numbers. If it's 1, each line will be numbered
	%stringstyle=\color{mymauve},     % string literal style
	identifierstyle=\color{mymauve},
	tabsize=2                       % sets default tabsize to 2 spaces
}

\setcounter{tocdepth}{1}


\begin{document}

\begin{titlepage}
\begin{center}

\hfill \break
\hfill \break

\includegraphics[width=0.3\textwidth]{./logo}~\\[1cm]

\textsc{\LARGE Instituto Superior Técnico}\\[0.25cm]
\textsc{\Large Mestrado Integrado em Engenharia Electrotécnica e de Computadores}\\[1.8cm]
\textsc{\huge Arquiteturas Avançadas de Computadores}\\[0.25cm]

{\huge \bfseries Paralelização e aceleração de um programa\\[2cm]}

\begin{tabular}{ l l }
Guilherme .....
Maria Margarida Dias dos Reis & \hspace{2mm} n.º 73099 \\
Nuno Miguel Rodrigues Machado & \hspace{2mm} n.º 74236
\end{tabular}

\vfill

{\large Lisboa, 3 de Maio de 2015} 

\end{center}
\end{titlepage}

\pagenumbering{gobble}
\clearpage

\tableofcontents
\pagebreak

\clearpage
\pagenumbering{arabic}

\section{Introdução}

\section{Implementação no CPU }
Inicialmente o algoritmo proposto foi implementado para correr só no CPU. Para isso, foi transcrito para o código C.
O algoritmo divide-se em duas partes fundamentais:


\begin{itemize}
	\vspace{-3mm}
	\item Criação dos dados de entrada;
	\vspace{-1.5mm}
	\item  Obtenção do \textit{Smooth} com processamento dos dados de entrada;
\end{itemize}
	\subsection{Criação dos dados de entrada}
	O sinal a ser processado é composto pela soma de dois sinais sinusoidais mais um erro com uma amplitude máxima de $0,1$. 
	Todos os calculos são feitos em \texttt{float}, em primeiro lugar é realizado a alocação da memória de todas as variáveis necessárias para o calculo do sinal de entrada. De seguida está representado o código C em detalhe.
\begin{lstlisting}[language=C]
#define N 10000;
	...

	int main() {
	
	float *x, *y, *yest_cpu, *randomArray;
	...
	
	/*Alocacao de memoria*/
	x = (float *)malloc(N*sizeof(float));
	y = (float *)malloc(N*sizeof(float));
	yest_cpu = (float *)malloc(N*sizeof(float));
	randomArray = (float *)malloc(N*sizeof(float));
	...
	exit(0);
	}
\end{lstlisting}

	A implementação do algoritmo do sinal de entrada é composta por um ciclo que itera o numero de amostras do sinal pretendido. Em primeiro lugar é necessário gerar os valores ao sinal que vai ser processado, estes valores são gerados pela seguinte equação
	\vspace{-3mm}
	\begin{equation}
		X ~= ~i/10; 
	\end{equation}
	de seguida é gerado um valor aleatório entre 1 e -1, simulando o ruído resultante da
	amostragem do sinal. Este valor é gerado pela função \texttt{randn()}, o código da função está representado de seguida, é de salientar que o código foi obtido da Internet, onde este simula a função \texttt{randn()} do MatLab :
	\begin{lstlisting}[language=C]
	float randn()
	{
		float x1, x2, w, y1;
		do
		{
			x1 = (float)(2.0 * rand() / RAND_MAX - 1.0);
			x2 = (float)(2.0 * rand() / RAND_MAX - 1.0);
			w = x1 * x1 + x2 * x2;
		} while (w >= 1.0);
	
		w = (float)sqrt((-2.0 * log(w)) / w);
		y1 = x1 * w;
		return y1;
	}
	\end{lstlisting}
	 Depois de obter os valores de \texttt{X} e do valor aleatório pode-se iterar os valores das amostras do sinal a ser processado. O código de seguida representa o ciclo que itera as amostras de \texttt{X}, do valor aleatório, \texttt{randomArray} e o sinal a ser processado, \texttt{Y}.
	 \begin{lstlisting}[language=C]
	 int main(){
		 for (int i = 0; i < N; ++i) {
			 x[i] = (float)i / 10;
			 randomArray[i] = randn();
			 y[i] = function((float)x[i], (float)randomArray[i]);
		 }
	...
	exit(0);
	}
	 \end{lstlisting}
	 
	
	\subsection{Obtenção do \textit{Smooth} com processamento dos dados de entrada}
	
	O algoritmo de \textit{Smooth} para o anulamento do ruído resultante da amostragem do sinal é aplicado segundo a expressão seguinte:
	\vspace{-3mm}
	\begin{equation}
		y_{est}~=~\sum_{i = 0}^{N-1}~\frac{\sum_{k = 0}^{N-1}Kb(x_i,x_k)y_k}{\sum_{k = 0}^{N-1}Kb(x_i,x_k)};
	\end{equation}
	\vspace{-3mm}
	\begin{equation}
	K_b(x_i,x_k)~=~e^{-\frac{(x_i - x_k)^2}{2 b^2}};
	\end{equation}
	A implementação do algoritmo em C baseia-se na utilização de um ciclo para o somatório exterior e outro ciclo para o somatório interior, as funções exponencial \texttt{expf} e potencia de base 2, \texttt{powf}, pertence á biblioteca, \texttt{math.h}. O código seguinte demonstra a utilização dos dois ciclos como tambem a das funções para o calculo do \textit{smoothing}:
	\begin{lstlisting}[language=C]
	int main(){
		float sumA, sumB;
		...
		
		for (int i = 0; i < N; ++i) { //percorrer o yest
			sumA = 0;
			for (int j = 0; j < N; ++j) { //percorer o input dataset
				sumA = sumA + ((expf(-powf((x[i] - x[j]), 2) / (2 * powf(SMOOTH, 2)))) * y[j]);
			}
			sumB = 0;
			for (int j = 0; j < N; ++j)	{ //percorer o input dataset
				sumB = sumB + expf(-powf((x[i] - x[j]), 2) / (2 * powf(SMOOTH, 2)));
			}
			yest_cpu[i] = sumA / sumB;
		}
	
		...
		exit(0);
	}
	\end{lstlisting}
	
	
\section{Implementação no GPU}
A implementação em GPU é dividia em 4 partes:
\begin{itemize}
	\vspace{-3mm}
	\item Alocação da memória;
	\vspace{-1.5mm}
	\item  Envio dos dados do CPU para o GPU;
	\vspace{-1.5mm}
	\item  Iniciação do algoritmo em GPU;
	\vspace{-1.5mm}
	\item  Envio dos dados do GPU para o CPU;
\end{itemize}

\subsection{Alocação da memória}
No inicio da implementação da paralisação em CUDA é necessário alocar a memória total a ser enviada do CPU para o GPU. Neste caso também foi alocado a memória total necessária para guardar o resultado do \textit{smoothing}. De seguida apresenta-se o código para alocação da memória do \textit{device} que com tem o GPU:
	\begin{lstlisting}[language=C]
	int main(){
	float *d_x, *d_y, *d_yest;
	...
	cudaMalloc(&d_x, N*sizeof(float)); 
	cudaMalloc(&d_y, N*sizeof(float));
	cudaMalloc(&d_yest, N *sizeof(float));
	...
	exit(0);
	}
	\end{lstlisting}
 
\subsection{Envio dos dados do CPU para o GPU}
Com a memória alocada , o passo seguinte é transferir os dados obtidos na secção 2.1, \texttt{X} e \texttt{Y}, para o \textit{device}. É utilizado a função \texttt{cudaMemcpy}, função que pertence à biblioteca \texttt{cuda.h},
\subsection{Iniciação do algoritmo em GPU}
\subsection{Envio dos dados do GPU para o CPU}
	
\section{Técnicas de aceleração e otimização}

\section{Secção de resultados}
\subsection{Tempos}
\subsection{Resultados do \textit{Smooth}}

\section{Conclusões}

\end{document}